\section{Results and Implementation}
The relevant figures have been attached and can be seen in the next page.

\subsection{Point Cloud Quality}
The reconstruction quality improved significantly through our workflow:

\begin{enumerate}
    \item \textbf{Initial SfM attempt}: Produced a sparse, low-quality point cloud with many gaps and inconsistencies
    
    \item \textbf{MetaShape on original images}: Generated a much better point cloud, but still had some areas with poor coverage
    
    \item \textbf{MetaShape on expanded image set}: The additional images provided better scene coverage and more redundancy for reconstruction. The final point cloud was significantly better in density and consistency
\end{enumerate}

\subsection{Virtual Tour Application}
The Three.js application successfully displays the point cloud and allows interactive navigation. The application features:

\begin{itemize}
    \item Real-time 3D rendering of the point cloud
    \item Mouse and keyboard controls for camera movement
    \item Smooth interaction on our desktop system
    \item Responsive zoom and rotation controls
\end{itemize}

\begin{figure}
    \centering
    \includegraphics[width=0.2\textwidth]{sections/figures/feature_matches.png}
    
    (a) Feature Mapping
    
    \includegraphics[width=0.2\textwidth]{sections/figures/two_view_cloud.png}

    (b) Two-view reconstruction

    \includegraphics[width=0.2\textwidth]{sections/figures/final_sfm_cloud.png}
    
    (c) MetaShape point cloud 
    
    \includegraphics[width=0.2\textwidth]{sections/figures/model.png}
    
    (d) MetaShape 3D model
    
    \caption{Reconstruction pipeline: (a) SIFT feature matches between consecutive images. (b) Sparse point cloud from two-view reconstruction. (c) Final point cloud after MetaShape processing on expanded image set. (d) 3D model generated by MetaShape.}
    \label{fig:reconstruction_results}
\end{figure}

\subsection{Visualization Quality}
We also examined the reconstructed scene from different perspectives. However, we encountered an issue with mirror reflections in the 3D viewer:

\begin{figure}
    \centering
    \includegraphics[width=0.2\textwidth]{sections/figures/mirrors.png}
    \caption{Virtual tour visualization showing the reconstructed point cloud. Mirror surfaces are not rendered correctly in the 3D viewer; reflective surfaces appear distorted or fragmented. This is likely due to the point cloud representation capturing mirror reflections as separate geometry rather than coherent surfaces, a known challenge in photogrammetry of reflective materials.}
    \label{fig:mirrors_visualization}
\end{figure}